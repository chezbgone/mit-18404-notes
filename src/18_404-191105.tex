\documentclass{standalone}
\usepackage{chez}

\begin{document}
%\chapter{November 05, 2019}
\section{Space Complexity}
Similar to time complexity, we can define space complexity.
\begin{definition}
  Given a function \(f \colon \NN \to \NN\) (assuming \(f(n) \geq n\) for convenience), we say that a \textsf{TM} \(M\) \vocab{runs in space \(f(n)\)} if \(M\) is a decider and uses \(\leq f(n)\) tape cells on all inputs of length \(n\).
  
  Similarly, a \textsf{NTM} \(M\) \vocab{runs in space \(f(n)\)} if \(M\) is a decider and uses \(\leq f(n)\) tapes cells on each branch for all inputs of length \(n\).

  \tcblower

  Then, we can define
  \begin{gather*}
    \operatorname{SPACE}(f(n)) = \set{
      A \mid \text{some \textsf{TM} decides $A$ in $O(f(n))$ time}
    } \\
    \operatorname{NSPACE}(f(n)) = \set{
      A \mid \text{some \textsf{NTM} decides $A$ in $O(f(n))$ time}
    }.
  \end{gather*}
\end{definition}

Then, similar to time complexity we can define
\begin{gather*}
  \textsf{PSPACE} = \bigunion_k \operatorname{SPACE}(n^k) \\
  \textsf{NPSPACE} = \bigunion_k \operatorname{NSPACE}(n^k).
\end{gather*}

For \(f(n) \geq n\), it is immediate that \(\operatorname{TIME}(f(n)) \subseteq \operatorname{SPACE}(f(n))\). This is because if a machine runs in time \(O(f(n))\), then it uses at most \(O(f(n))\) space as well, because the machine can only use one tape cell per time step. This implies that \(\mathsf{P} \subseteq \mathsf{PSPACE}\), and for a similar reason, we have \(\mathsf{NP} \subseteq \mathsf{NPSPACE}\).

Conversely, we have \(\operatorname{SPACE}(f(n)) \subseteq \operatorname{TIME}(2^{O(f(n))})\), because if a machine uses \(O(f(n))\) space and must halt, then it runs in time \(c \cdot \size{\Gamma}^{O(f(n))}\) for some constant \(c\).

\begin{theorem}
  \(\mathsf{NP} \subseteq \mathsf{PSPACE}\).
\end{theorem}
\begin{proof}
  Note that \(\textit{SAT} \in \mathsf{PSPACE}\) because we can keep track of the formula and an assignment in polynomial space, and test each possible assignment. Then, for \(A \in \mathsf{NP}\), we know \(A \polyredu SAT\), so \(A \in \mathsf{PSPACE}\).
  % why not do this directly?
\end{proof}

\begin{example}
  We have \(\mathsf{coNP} = \set{\ol A \mid A \in \mathsf{NP}} \subseteq \mathsf{PSPACE}\). This is because 

  Some examples of problems in \(\mathsf{coNP}\) are the unsatisfiability problem, or tautology problem
  \begin{gather*}
    \textit{UNSAT} = \set{\angles \phi \mid \text{formula $\phi$ is not satisfied for some assignment}} \\
    \textit{TAUT} = \set{\angles \phi \mid \text{formula $\phi$ is satisfied for all assignment}}.
  \end{gather*}
\end{example}

\subsection{Quantified Boolean Formulas}
Let a \vocab{quantified boolean formula} be a boolean formula with quantifiers \(\forall\) and \(\exists\).
\begin{example}
  The two formulas
  \begin{gather*}
    \forall x \exists y [(x \lor y) \land (\ol x \lor \ol y)] \\
    \forall y \exists x [(x \lor y) \land (\ol x \lor \ol y)]
  \end{gather*}
  are quantified boolean formulas that are \textsc{true} and \textsc{false} respectively.
\end{example}

Consider the problem of deciding whether a quantified boolean formula is true or not, i.e.
\[
  \textit{TQBF} = \set{\angles\phi \mid \text{$\phi$ is a \textsc{true} quantified boolean formula}}.
\]
The \(\textit{T}\) in \(\textit{TQBF}\) stands for true.

\begin{proposition}<tqbf_pspace>
  \(\textit{TQBF} \in \mathsf{PSPACE}\).
\end{proposition}
\begin{proof}
  Consider the following algorithm:
  \begin{enumerate}[start=0]
    \item On input \(\angles \phi\):

    \item If there are no quantifiers, then there are only constants, so we can evaluate \(\phi\) directly and \textsc{accept} if \(\phi\) is true.

    \item If \(\phi = \exists x \psi\) for some quantified boolean formula \(\psi\), then recursively evaluate \(\psi\) with \(x \in \set{\textsc{true}, \textsc{false}}\). If either case accepts, then \textsc{accept}.

    \item If \(\phi = \forall x \psi\) for some quantified boolean formula \(\psi\), then recursively evaluate \(\psi\) with \(x \in \set{\textsc{true}, \textsc{false}}\). If both cases accept, then \textsc{accept}.
  \end{enumerate}
  Note that this algorithm can be run in polynomial space because there are at most \(n\) recursive calls, and each recursive call uses \(O(1)\) space.
\end{proof}

\subsection{Word Ladder}
There are some games where you have two words and try to get from one to the other by changing one letter at a time, while remaining on valid words. E.g.
\[
  \text{WORK} \to
  \text{PORK} \to
  \text{PORT} \to
  \text{SORT} \to
  \text{SOOT} \to
  \text{SLOT} \to
  \text{PLOT} \to
  \text{PLOY} \to
  \text{PLAY}.
\]
Let's define the problem
\[
  \begin{multlined}
    \textit{LADDER}_{\textsf{DFA}} = \lbrace
      \angles{B, u, v} \mid \text{$B$ is a \textsc{DFA} such that there exists a sequence $y_0, \dots, y_k$ in $L(B)$} \\
      \text{such that $y_0 = u$, $y_1 = v$, and $y_i$ differs from $y_{i + 1}$ in one symbol}
    \rbrace.
  \end{multlined}
\]

\begin{proposition}
  \(\textit{LADDER}_{\textsf{DFA}} \in \mathsf{NPSPACE}\).
\end{proposition}
\begin{proof}
  Consider the following nondeterministic algorithm:
  \begin{enumerate}[start=0]
    \item On input \(\angles{B, u, v}\):
    \item Make a copy of \(u\), called \(y\).
    \item Nondeterministically change one letter.
    \item Repeat either until \(y = v\) (in which case \textsc{accept}), or \(\size{u}^{\size{\Sigma}}\) letters have been changed (in which case \textsc{reject}), where \(\Sigma\) is the alphabet of \(B\).
  \end{enumerate}
  Note that this algorithm runs in polynomial space for all branches, so \(\textit{LADDER}_{\textsf{DFA}} \in \mathsf{NPSPACE}\).
\end{proof}




\end{document}

