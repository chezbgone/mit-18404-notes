\documentclass{standalone}
\usepackage{chez}

\begin{document}
\section{More on Regular Expressions}
\begin{proposition}
	The set of languages that regular expressions determine is
  exactly all regular languages.
\end{proposition}
\begin{proof}[Regular expressions \(\implies\) regular]
	It is fairly clear that all regular expressions determine regular languages.
  In particular, for the atomic regular expressions,
  we can find the \textsf{NFA}s:
	\begin{itemize}
		\item \(a\) for \(a \in \Sigma_\eps\)
		\begin{center}
			\begin{tikzpicture}[automaton]
				\node[state, initial]                (q0) {};
				\node[state, accepting, right=of q0] (q1) {};
				\path (q0) edge node{\(a\)} (q1);
			\end{tikzpicture}
		\end{center}

		\item \(\nullset\)
		\begin{center}
			\begin{tikzpicture}[automaton]
				\node[state, initial] (q0) {};
			\end{tikzpicture}
		\end{center}
	\end{itemize}

	For the composite regular expressions,
  we just can use the regular operations for \textsf{NFA}s.
\end{proof}

\begin{example}
	If we want to build an \textsf{NFA} for \((\mathtt a \union \mathtt{ab})^*\),
  we can just build it up from smaller parts:
	\begin{itemize}[nosep]
		\item \(\mathtt a\)
		\begin{center}
			\begin{tikzpicture}[automaton]
				\node[state, initial]                (q0) {};
				\node[state, accepting, right=of q0] (q1) {};
				\path (q0) edge node{\(\mathtt a\)} (q1);
			\end{tikzpicture}
		\end{center}

		\item \(\mathtt b\)
		\begin{center}
			\begin{tikzpicture}[automaton]
				\node[state, initial]                (q0) {};
				\node[state, accepting, right=of q0] (q1) {};
				\path (q0) edge node{\(\mathtt b\)} (q1);
			\end{tikzpicture}
		\end{center}

		\item \(\mathtt{ab}\)
		\begin{center}
			\begin{tikzpicture}[automaton]
				\node[state, initial]                (q0) {};
				\node[state, right=of q0]            (q1) {};
				\node[state, right=of q1]            (q2) {};
				\node[state, accepting, right=of q2] (q3) {};
				\path (q0) edge node{\(\mathtt a\)} (q1);
				\path (q1) edge node{\(\eps\)} (q2);
				\path (q2) edge node{\(\mathtt b\)} (q3);
			\end{tikzpicture}
		\end{center}

		\item \(\mathtt a \union \mathtt{ab}\)
		\begin{center}
			\begin{tikzpicture}[automaton]
				\node[state, initial]                (qn1) {};
				\node[state, below right=0.6cm and 1cm of qn1]     (q0) {};
				\node[state, right=of q0]            (q1) {};
				\node[state, right=of q1]            (q2) {};
				\node[state, accepting, right=of q2] (q3) {};
				\node[state, above right=0.6cm and 1cm of qn1]     (q4) {};
				\node[state, accepting, right=of q4] (q5) {};
				\path (qn1) edge node[swap]{\(\mathtt \eps\)} (q0);
				\path (qn1) edge node{\(\mathtt \eps\)} (q4);
				\path (q0) edge node{\(\mathtt a\)} (q1);
				\path (q1) edge node{\(\eps\)} (q2);
				\path (q2) edge node{\(\mathtt b\)} (q3);
				\path (q4) edge node{\(\mathtt a\)} (q5);
			\end{tikzpicture}
		\end{center}

		\item \((\mathtt a \union \mathtt{ab})^*\)
		\begin{center}
			\begin{tikzpicture}[automaton]
				\node[state, initial, accepting]     (qn2) {};
				\node[state, right=of qn2]           (qn1) {};
				\node[state, below right=0.6cm and 1cm of qn1]     (q0) {};
				\node[state, right=of q0]            (q1) {};
				\node[state, right=of q1]            (q2) {};
				\node[state, accepting, right=of q2] (q3) {};
				\node[state, above right=0.6cm and 1cm of qn1]     (q4) {};
				\node[state, accepting, right=of q4] (q5) {};

				\path (qn2) edge node{\(\mathtt \eps\)} (qn1);
				\path (qn1) edge node[swap]{\(\mathtt \eps\)} (q0);
				\path (qn1) edge node{\(\mathtt \eps\)} (q4);
				\path (q0) edge node{\(\mathtt a\)} (q1);
				\path (q1) edge node{\(\eps\)} (q2);
				\path (q2) edge node{\(\mathtt b\)} (q3);
				\path (q4) edge node{\(\mathtt a\)} (q5);
				\path[bend right=45] (q5) edge node[swap]{\(\eps\)} (qn2);
				\path[bend left=30] (q3) edge node{\(\eps\)} (qn2);
			\end{tikzpicture}
		\end{center}
	\end{itemize}
\end{example}

To prove that all \textsf{NFA}s can be represented by a regular expression,
we will define another version of them that will make things more convenient.
\begin{definition}
	A \vocab{generalized nondeterministic finite automaton (\textsf{GNFA})}
  is an \textsf{NFA} that can have regular expressions as transition labels.
\end{definition}

\begin{example}
	This is a \textsf{GNFA}, and it accepts \(\mathtt{abbbabab}\):
	\begin{center}
		\begin{tikzpicture}[automaton, node distance=2cm]
			\node[state, initial] (q0) {};
			\node[state, right=of q0] (q1) {};
			\node[state, accepting, below right=of q0] (q2) {};

			\path (q0) edge node{\(\mathtt{ab}^*\)} (q1);
			\path[loop above] (q1) edge node{\(\mathtt{bb}\)} (q1);
			\path (q0) edge node[swap]{\(\eps\)} (q2);
			\path (q1) edge node[pos=0.3]{\((\mathtt a \union \mathtt{ab})^*\)} (q2);
		\end{tikzpicture}
	\end{center}
\end{example}

\begin{proof}[\textsf{GNFA} \(\implies\) regex]
	Suppose we have a \textsf{GNFA}.
  For convenience, assume that the starting state is not accepting.
  (We can just add another starting state with an \(\eps\) arrow
  pointing to it otherwise.)
  Also, assume that there are only arrows out of the initial state,
  and there are only arrows going into a single accepting state.
  We can ensure this by adding \(\eps\) arrows.

	We proceed by induction.
  The smallest machine that satisfies these conditions
  is one that has \(k = 2\) states, and it looks like this
	\begin{center}
		\begin{tikzpicture}[automaton]
			\node[state, initial] (q0) {};
			\node[state, accepting, right=of q0] (q1) {};
			\path (q0) edge node{\(r\)} (q1);
		\end{tikzpicture}
	\end{center}
	where \(r\) is some regular expression.
  This is the regular expression for this \textsf{GNFA}.

    Now suppose that all \textsf{GNFA}s with at most \((k-1)\) states
    has an equivalent regular expression.
    Consider some \textsf{GNFA} \(G\) with \(k\) states.
    Our goal is to make a new \textsf{GNFA} with \(k - 1\) states.
    Let's remove some state \(g_x\).
    For all states \(q_i, q_j \neq g_x\) in \(G\),
    we an make the following change to fix the machine:
	\begin{center}
		\begin{tikzpicture}[automaton]
			\node[state] (qi) {\(q_i\)};
			\node[state, right=2cm of qi] (qj) {\(q_j\)};
			\node[state] (qx) at ($(qi)!0.5!(qj) + (0, 1.4)$) {\(q_x\)};

			\path (qi) edge node{\(r_1\)} (qx);
			\path[loop above] (qx) edge node{\(r_2\)} (qx);
			\path (qx) edge node{\(r_3\)} (qj);
			\path (qi) edge node[swap]{\(r_4\)} (qj);

			\draw[lightgray] ($(qi) + (-0.6, -0.6)$) rectangle ($(qx) + (1.6, 1.3)$);
			
			\node (arr) at ($(qx) + (2.4, -0.3)$) {\(\implies\)};

			\node[state] (qip) at ($(arr) + (1.4, 0)$) {\(q_i\)};
			\node[state, right=3cm of qip] (qjp) {\(q_j\)};
			\path (qip) edge node{\(r_1 (r_2)^* r_3 \union r_4\)} (qjp);
			\draw[lightgray] ($(qip) + (-0.6, -0.6)$) rectangle ($(qjp) + (0.6, 0.7)$);
		\end{tikzpicture}
	\end{center}
	This gives a machine with \(k - 1\) states,
  and it accepts the same language, as desired.
\end{proof}

\section{Non-regularity}
Let \(B = \set{w \mid \text{$w$ has an equal number of
                            $\mathtt a$s and $\mathtt b$s}}\) and
    \(C = \set{w \mid \text{$w$ has an equal number of
                            $\mathtt{ab}$s and $\mathtt{ba}$s substrings}}\).
It turns out that \(B\) is not regular but \(C\) is regular.
It's not obviously clear how to determine this,
so it would be nice to have a method to determine
whether a language is regular or not.

\begin{proposition}[Pumping Lemma]
If \(A\) is a regular language, there exists some \(p \in \NN\) called
the \vocab{pumping length} such that if \(s \in A\) with \(\size s \geq p\),
we can write \(s = xyz\) where
\begin{itemize}[nosep]
\item \(xy^i z \in A\) for \(i \in \set{0, 1, 2, \dots}\),
\item \(y \neq \eps\), and
\item \(\size{xy} \leq p\).
\end{itemize}
\end{proposition}
\begin{proof}
	Let \(M\) be a \textsf{DFA} for a regular language \(A\),
  and let \(p\) be the number of states of \(M\).
  Let \(s \in A\) have length \(\size s \geq p\).
	
	Consider the states that the machine will visit
  while going through the string.
  Since \(\size s \geq p\), the machine will visit at least \(p + 1\) states.
  By the pigeonhole principle, the machine will visit
  some state \(q\) at least twice.
	\begin{center}
		\begin{tikzpicture}[automaton,
                        decoration={snake, amplitude=0.5mm, post length=1mm}]
			\node[state, initial] (q0) {};
			\node[state, above right=of q0] (q1) {\(q\)};
			\node[state, accepting, below right=of q1] (q2) {};
			
			\path (q0) edge[decorate] node{\(x\)} (q1);
			\path (q1) edge[decorate] node{\(y\)} (q2);
			\path (q1) edge[decorate, in=40, out=140, looseness=20]
                 node[above=0.1]{\(z\)} (q1);
		\end{tikzpicture}
	\end{center}
	Then, that means there is a nonempty string \(y\) that transitions from state \(q\) to \(q\), so we can repeat is as many times as possible.
\end{proof}

We can now use the pumping lemma to prove things!
\begin{claim}
	Let \(D = \set{\mathtt a^k \mathtt b^k \mid k \in \NN}\).
  Then \(D\) is not regular.
\end{claim}
\begin{proof}
	Assume for the sake of contradiction that \(D\) is regular.
  Then let the pumping length be \(p\), and consider
  \(s = \mathtt a^p \mathtt b^p \in D\).
  By the pumping lemma, we can write \(s = xyz\)
  where \(y = \mathtt a^n\) for some \(n > 0\).
  Then, this gives \(\mathtt a^{p + kn}\mathtt b^{p} \in D\)
  for all \(k \in \set{0, 1, \dots}\), which is a contradiction.
\end{proof}




\end{document}

