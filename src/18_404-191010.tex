\documentclass{standalone}
\usepackage{chez}

\begin{document}
\section{More Undecidability}
\begin{proposition}
  The language \(\mathit{ALL}_{\textsf{CFG}} = \set{
    \angles G \mid \text{\textsf{CFG} $G$ such that $L(G) = \Sigma^*$}
  }\) is undecidable.
\end{proposition}
\begin{proof}
  We will reduce \(A_{\textsf{TM}}\) to \(\textit{ALL}_{\textsf{CGF}}\) using computation history. The idea is to construct a \textsf{PDA} whose language is all strings except the accepting computation history for \(M\) on \(w\).

  Suppose that the \textsf{TM} \(R\) decides \(\textit{ALL}_{\textsf{CFG}}\). Consider the following \textsf{TM} \(S\) that decides \(A_{\textsf{TM}}\):
  \begin{enumerate}[start=0]
    \item On input \(\angles{M, w}\):
    \item Construct the \textsf{PDA} \(P_{M, w}\) that does the following:
    \begin{enumerate}[(1), start=0, nosep]
      \item On input \(C_1 \texttt\# C_2^{\mathcal R} \texttt\# C_3 \texttt\# C_4^{\mathcal R} \texttt\# \dots \texttt\# C_k\):
      \item Nondeterministically accept if any of the following hold:
      \begin{itemize}[nosep]
        \item \(C_1\) is not the starting configuration of \(M\) on \(w\),
        \item \(C_k\) is not an accepting state,
        \item for some \(i\), \(C_i\) does not yield \(C_{i + 1}\).
      \end{itemize}
    \end{enumerate}
    \item Convert \(P_{M, w}\) to \(G_{M, w}\).
    \item Run \(R\) on \(G_{M, w}\).
    \item If \(R\) accepts then \textsc{reject}, if \(R\) rejects, then \textsc{accept}.
  \end{enumerate}

  The way that \(P_{M, w}\) works is that on each branch of nondeterminism, \(P_{M, w}\) checks each condition for the input to be a valid computation history. To check that \(C_i\) yields \(C_{i + 1}\) for each \(i\), we require the input to have every other \(C_i\) be reversed, so that we can compare adjacent \(C_i\)s.
\end{proof}

\section{Recursion Theorem}
Sometimes we think about machines that can construct replicas of themselves. We see this in biology! In fact, in computing, there exists a \textsf{TM} \(\mathit{SELF}\) that prints out its own description \(\angles{\textit{SELF}}\).  To show that we can do this, we need a lemma.

\begin{lemma}
  There exists a computable function \(q \colon \Sigma^* \to \Sigma^*\) mapping \(w \mapsto \angles{P_w}\), where \(P_w\) prints \(w\).
\end{lemma}
\begin{proof}
  This is straightforward.
\end{proof}

Now to construct \(\mathit{SELF}\), suppose that it is composed of two \textsf{TM}s \(A\) and \(B\) that compute in that order. Let \(A\) print \(P_{\angles B}\). Then, let \(B\) compute the function \(q\) as described above on the contents of the tape (which is \(P_{\angles B}\)), giving \(\angles A\). Then, prepend \(\angles A\) to the tape. This means that we \(\mathit{SELF}\) prints \(\angles A \angles B\), which is just what we want.

We can even implement this in English!
\begin{algorithm*}
  Print out two copies of the following, the second copy in quotes.

  ``Print out two copies of the following, the second copy in quotes.''
\end{algorithm*}

This gives us the following theorem:
\begin{theorem}
  A \textsf{TM} can obtain its own description.
\end{theorem}

\begin{corollary}
  The language \(A_{\textsf{TM}}\) is undecidable.
\end{corollary}
\begin{proof}
  Suppose for the sake of contradiction that \(H\) decides \(A_{\textsf{TM}}\). Then construct the \textsf{TM} \(R\) as follows:
  \begin{enumerate}[start=0]
    \item On input \(w\):
    \item Get its own description \(\angles{R}\).
    \item Run \(H\) on \(\angles{R, w}\).
    \item Output the opposite of what \(H\) outputs.
  \end{enumerate}
  
  This is a contradiction, because if we consider some input \(w\), \(R\) cannot accept nor reject, as a contradiction would arise in both cases.
\end{proof}

\begin{proposition}
  The language \(\mathit{MIN}_{\textsf{TM}} = \set{
    \angles{M} \mid \text{$\angles M$ is the shortest description among all equivalent \textsf{TM}s}
  }\) is unrecognizable.
\end{proposition}
\begin{proof}
  Suppose for the sake of contradiction that \(\mathit{MIN}_{\textsf{TM}}\) is recognizable. Then some Turing enumerator \(E\) enumerates \(\mathit{MIN}_{\textsf{TM}}\). Consider the following \textsf{TM} \(C\):
  \begin{enumerate}[start=0]
    \item On input \(w\):
    \item Get \(\angles{C}\).
    \item Run \(E\) until we get some description \(\angles D\) longer than \(\angles C\).
    \item Output what \(D\) outputs when run on \(w\).
  \end{enumerate}

  This gives a contradiction, because \(D\) cannot be minimal, as \(C\) does the same thing as \(D\).
\end{proof}



\end{document}

