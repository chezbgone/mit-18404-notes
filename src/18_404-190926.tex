\documentclass{standalone}
\usepackage{chez}

\begin{document}
\section{Decision Problems}
Sometimes, we want to work with objects that are not strings.
If \(A\) is an object, then let's say that \(\angles A\)
is an \vocab{encoding} of \(A\) into a string over \(\Sigma\).
If we have multiple objects \(A_1, \dots, A_k\),
then we can encode all of them into one string
using \(\angles{A_1, \dots, A_k}\).
Note that \(\angles{A_1, \dots, A_k}\) is
not necessarily just \(\angles{A_1}, \dotsc, \angles{A_k}\) concatenated,
because it might not be clear where each one ends.

\subsection{DFAs}
\subsubsection{Acceptance Problem}
Given a \textsf{DFA}, we would like to determine
whether it will accept some string \(w\).
In particular, let
\[
	A_{\textsf{DFA}} = \set{
		\angles{B, w} \mid
		\text{$B$ is a \textsf{DFA} that accepts $w$}
	}.
\]
Therefore, if testing if \(a \in A\) is equivalent to
determining if some \textsf{DFA} accepts some word.

\begin{proposition}
	\(A_\textsf{DFA}\) is decidable.
\end{proposition}
\begin{proof}
	Let \(M\) be the following Turing machine with input \(s = \angles{B, w}\):
	\begin{itemize}[nosep]
		\item Test if \(s = \angles{B, w}\) rejects or not.
		\item Simulate \(B\) on \(W\).
		\item If \(B\) accepts, \textsc{accept}, else \textsc{reject}. \qedhere
	\end{itemize}
\end{proof}

We can define a similar language for \textsf{NFA}s.
\begin{corollary}
	\(A_{\textsf{NFA}} = \set{\angles{B, w} \mid
		\text{$B$ is a \textsf{NFA} that accepts $w$}
	}\) is decidable.
\end{corollary}
To prove this, we can just do the same thing as above,
and simulate the \textsf{NFA}.
However, we can do something clever and just reduce it to the previous problem.
\begin{proof}[Sketch]
	Convert \(B\) to a \textsf{DFA} \(B'\) and then
  use the fact that \(A_{\textsf{DFA}}\) is decidable.
\end{proof}

\subsubsection{Emptiness Problem}
\begin{proposition}
	Let \(E_{\textsf{DFA}} = \set{\angles{B} \mid
		\text{$B$ is a \textsf{DFA} where $L(B) = \nullset$}
	}\). Then \(E_{\textsf{DFA}}\) is decidable.
\end{proposition}
\begin{proof}
	The idea is to just perform a graph search on the states.
  Consider the \textsf{TM} \(M\) with input \(\angles B\), described by:
	\begin{itemize}
		\item Mark the start state
		\item Repeat until nothing new is marked:
		\begin{itemize}[nosep]
			\item Mark state \(q\) if some previously marked state points to \(q\).
		\end{itemize}
		\item \textsc{Accept} if no accept node is marked,
          otherwise \textsc{reject}. \qedhere
	\end{itemize}
\end{proof}

\subsubsection{Equivalence Problem}
\begin{proposition}
	Let \(\mathit{EQ}_{\textsf{DFA}} = \set{
		\angles{B, C} \mid \text{$B, C$ are \textsf{DFA}s and $L(B) = L(C)$}
	}\). Then \(EQ_{\textsf{DFA}}\) is decidable.
\end{proposition}
\begin{proof}
	The key idea is that if \(L(B) = L(C)\),
  then we know that the symmetric difference is empty, i.e.
	\[
		L(B) \mathbin{\Delta} L(C)
      = \ol{L(B)} \intersect L(C) \union L(B) \intersect \ol{L(C)}
      = \nullset.
	\]
	Therefore, we consider the following \textsf{TM}
  with input \(\angles{B, C}\):
	\begin{itemize}
		\item Let \(D\) be the \textsf{DFA} such that
		\[
			L(D) = \ol{L(B)} \intersect L(C) \union L(B) \intersect \ol{L(C)}.
		\]
		\item Test if \(\angles{D} \in E_{\textsf{DFA}}\),
          i.e.\ if \(L(D)\) is empty.
          If yes then \textsc{accept}; \textsc{reject} if no. \qedhere
	\end{itemize}
\end{proof}

\subsection{CFGs}
\begin{proposition}
	Let \(A_{\textsf{CFG}} = \set{
		\angles{G, w} \mid \text{$G$ is a \textsf{CFG} such that $w \in L(G)$}
	}\). Then \(A_{\textsf{CFG}}\) is decidable.
\end{proposition}
\begin{proof}
	Consider the following \textsf{TM} with input \(\angles{G, w}\):
	\begin{itemize}
		\item Convert \(G\) to Chomsky's Normal form.
          We know that to derive a string of length $n$,
          it will take \(2n - 1\) steps
          (\(n - 1\) to get the correct number of symbols,
          and \(n\) to convert variables).
		\item Try all the derivations of length \(2n - 1\).
          If any of them yield \(w\), then \textsc{accept},
          else \textsc{reject}. \qedhere
	\end{itemize}
\end{proof}

\begin{proposition}
	Every \textsf{CFL} is decidable.
\end{proposition}
\begin{proof}
	Let \(A\) be a \textsf{CFL} generated by the \textsf{CFG} \(G\).
  Consider the \textsf{TM} \(M_G\) with input \(w\):
	\begin{itemize}
		\item Test if \(\angles{G, w} \in A_{\textsf{CFG}}\).
		\item If yes then \textsc{accept}; \textsc{reject} if no. \qedhere
	\end{itemize}
\end{proof}

\begin{proposition}
	Let \(E_{\textsf{CFG}} = \set{
		\angles{G} \mid \text{$G$ is a \textsf{CFG} and $L(G) = \nullset$}
	}\). Then \(E_{\textsf{CFG}}\) is decidable.
\end{proposition}
\begin{proof}
	Suppose we are given a list of rules.
  We can mark all the terminals,
  and then if everything on the right side of a rule is marked,
  then this means we can get the left side.
  Repeat until nothing new is marked.
  Then, if the initial variable is marked then \textsc{accept};
  else \textsc{reject}.
\end{proof}

\begin{proposition}
	Let \(\mathit{EQ}_{\textsf{CFG}} = \set{
		\angles{G, H} \mid \text{$G, H$ are \textsf{CFG}s and $L(G) = L(H)$}
	}\). Then \(\mathit{EQ}_{\textsf{CFG}}\) is undecidable.
\end{proposition}
We can't use the same trick of using the symmetric difference
we used last time for \textsf{DFA}s,
because \textsf{CFG}s are not closed under complements or intersection!
We'll prove this proposition later.

\begin{proposition}
	Let \(A_{\textsf{TM}} = \set{
		\angles{M, w} \mid \text{$M$ is a \textsf{TM} accepting $w$}
	}\). Then \(A_{\textsf{TM}}\) is Turing recognizable.
\end{proposition}
\begin{proof}
	The idea is to just make the machine do it.
  In particular, the following machine recognizes \(A_{\textsf{TM}}\):
	\begin{enumerate}[nosep, start=0]
		\item On input \(\angles{M, w}\):
		\item Simulate \(M\) on \(w\).
		\item If \(M\) accepts, then \textsc{accept},
          if \(M\) rejects, then \textsc{reject}. \qedhere
	\end{enumerate}
\end{proof}



\end{document}

