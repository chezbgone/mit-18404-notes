\documentclass{standalone}
\usepackage{chez}

\begin{document}
\section{CFGs and PDAs}
\begin{proposition}
	A language \(A\) is a \textsf{CFL} if and only if \(A = L(P)\) for some pushdown automata \(P\).
\end{proposition}
\begin{proof}[\(\textsf{CFG} \to \textsf{PDA}\)]
	The idea is that we can simulate a \textsf{CFG} using the stack to remember the intermediate strings. Initially, we insert the starting variable in the stack, and then when we see a variable at the top of a stack, we can replace it with any of the variable's rules. When we see a terminal at the top of the stack, we compare it with the input. If the \textsf{CFG} can generate the string, then eventually, we will remove everything in the stack. Overall, we have the following \textsf{PDA}:
	\begin{center}
		\begin{tikzpicture}[automaton, node distance=1.3cm]
			\node[state, initial]                (q0) {};
			\node[state, below=of q0]            (q1) {};
			\node[state, accepting, below=of q1] (q2) {};
			
			\path (q0) edge node{\(\eps, \eps \to \texttt \$ S\)} (q1);
			\path[loop right] (q1) edge
				node{\(\eps, \text{$A \to A'$ for all rules $A \to A'$}\)} (q1);
			\path[loop left] (q1) edge node{\(a, a \to \eps\)} (q1);
			\path (q1) edge node{\(\eps, \texttt \$ \to \eps\)} (q2);
		\end{tikzpicture}
	\end{center}
\end{proof}

\begin{proposition}[Pumping Lemma for \textsf{CFL}s]
	Given a \textsf{CFL} \(A\), there exists a \vocab{pumping length} \(p\) such that for all \(s \in A\) where \(\size s > p\), we can write \(s = uvxyz\) such that
	\begin{itemize}[nosep]
		\item \(uv^i x y^i z \in A\) for all \(i \in \set{0, 1, \dots}\),
		\item \(\size{vy} > 0\), and
		\item \(\size{vxy} \leq p\).
	\end{itemize}
\end{proposition}
\begin{proof}
	The idea is that given a \textsf{CFG} \(G\), there is some maximum rule length, so given some final string length, we can find a lower bound for the parse tree height. Therefore, for a sufficiently large final string length, we have a sufficiently tall parse tree.
	
	\begin{center}
		\begin{tikzpicture}[scale=0.9]
			\node[inner sep=0.15em] (R) at (0, 0) {\(R\)};
			\draw (R) -- +(2, -2)
				+(-2, -2) -- (R);

			\node[inner sep=0.15em] (R2) at (0.1, -1) {\(R\)};
			\draw (R2) -- +(1, -1)
				+(-1, -1) -- (R2);

			\draw ($(R) + (-2, -2)$) -- node[below]{\(v\)} ($(R2) + (-1, -1)$);
			\draw ($(R2) + (-1, -1)$) -- node[below]{\(x\)} ($(R2) + (1, -1)$);
			\draw ($(R2) + (1, -1)$) -- node[below]{\(y\)} ($(R) + (2, -2)$);

			\node[below right=0.7 and 2 of R] (arr) {\(\to\)};

			\node[inner sep=0.15em, above right=0.7 and 2 of arr] (R3) {\(R\)};
			\draw (R3) -- +(2, -2)
				+(-2, -2) -- (R3);

			\node[inner sep=0.15em] (R4) at ($(R3) + (0.1, -1)$) {\(R\)};
			\draw (R4) -- +(2, -2)
				+(-2, -2) -- (R4);
			\draw ($(R3) + (-2, -2)$) -- node[below]{\(v\)} ($(R4) + (-1, -1)$);
			\draw ($(R4) + (1, -1)$) -- node[below]{\(y\)} ($(R3) + (2, -2)$);

			\draw ($(R4) + (-2, -2)$) -- node[below]{\(vxy\)} ($(R4) + (2, -2)$);
		\end{tikzpicture}
	\end{center}

	If \(n = \size V\) is the number of variables, then consider a path from the initial variable to some terminal that has more than \(n\) variables. By pigeonhole, there exists some variable \(R\) that occurs more than once. Let the final result of the later \(R\) be \(x\), and the earlier \(R\) be \(vxy\). Then we can replace the later \(R\) with \(vxy\), and using a similar strategy, we can replace the earlier \(R\) with any \(v^i x y^i\).

	More formally, if \(b\) is the longest length of the right sides of all rules, then for a tree of height \(h\), the string length \(s\) satisfies \(s \leq b^h\). This means if we let \(p = b^{\size V + 1}\), we must have \(h \geq \size V + 1\), and there must be some repeated variable.

	Another thing we might need to worry about is if we have some rules of the form
	\[
		R \to S \to \dots \to R.
	\]
	To fix this, we can just say that given some string \(s \in A\), we will take the shortest parse tree for \(s\), i.e. we don't waste any time with rules that don't do anything.
\end{proof}

\begin{corollary}
	The language \(B = \set{\mathtt a^k \mathtt b^k \mathtt c^k}\) is not a \textsf{CFL}.
\end{corollary}
\begin{proof}
	For the sake of contradiction suppose \(B\) is regular. Then consider \(s = \mathtt a^p \mathtt b^p \mathtt c^p \in B\), where \(p\) is the pumping length. By the pumping lemma, we can write \(\mathtt a^p \mathtt b^p \mathtt c^p = uvxyz\) such that \(\size{vxy} \leq p\). In particular, this means that \(v\) and \(y\) can only contain two distinct letters. Therefore the pumping lemma states that \(uxz \in B\). However, we know that at least one of the number of \(\mathtt a\)s, \(\mathtt b\)s, and \(\mathtt c\)s must have changed, but not all of them, which is a contradiction.
\end{proof}

\begin{corollary}
	The language \(C = \set{ww \mid w \in \set{\mathtt a, \mathtt b}^*}\) is not context free.
\end{corollary}
\begin{proof}
	For the sake of contradiction, suppose that \(C\) is regular. Then consider \(s = \mathtt a^p \mathtt b^p \mathtt a^p \mathtt b^p \in C\), where \(p\) is the pumping length. By the pumping lemma, we can write \(s = uvwxy\) where \(\size{vwy} \leq p\). Therefore, \(vwy\) cannot contain \(\mathtt a\)s from both strings, and also cannot contain \(\mathtt b\)s from both strings. Therefore, when we pump, we will be changing the number of \(\mathtt a\)s (or \(\mathtt b\)s) in one of the \(w\)s but not the other.
\end{proof}

\chapter{Computability Theory}
\section{Turing Machines}
We would like a model of model of computation that is stronger than this. In particular, it will be similar to a pushdown automata, except instead of a stack, the machine reads and writes to a unbounded tape that initially contains the input, that it can move left and right on, and it accepts/rejects by entering special halting states.

\begin{example}
	Recall the language \(B = \set{\mathtt a^k \mathtt b^k \mathtt c^k}\). We claim that there is a Turing machine for \(B\).

	\tcblower
	The idea is that on the first pass (staring from the leftmost position), the machine tests to make sure that the input is of the form \(\mathtt a^* \mathtt b^* \mathtt c^*\). Then, after that we can make multiple passes, each one replacing one \(\mathtt a\), one \(\mathtt b\), and one \(\mathtt c\) with a blank space each time.
\end{example}

\begin{definition}
	A \vocab{Turing machine (\textsf{TM})} is a tuple \((Q, \Sigma, \Gamma, \delta, q_0, q_{\text{acc}}, q_{\text{reg}})\) such that
	\begin{itemize}[nosep]
		\item \(Q\) is the set of states,
		\item \(\Sigma\) is the input alphabet not containing \blank,
		\item \(\Gamma\) is the tape alphabet with \(\blank \in \Gamma\) and \(\Sigma \subseteq \Gamma\),
		\item \(\delta \colon Q \times \Gamma \to Q \times \Gamma \times \set{\mathrm L, \mathrm R}\) is the transition function,
		\item \(q_0 \in Q\) is the start state,
		\item \(q_{\text{accept}} \in Q\) is the accept state, and
		\item \(q_{\text{reject}} \in Q\) is the reject state, where \(q_{\text{accept}} \neq q_{\text{reject}}\).
	\end{itemize}
\end{definition}

The Turing machine begins its computation with the input on the leftmost part of the tape, and the rest of the tape is filled with \blank. The read/write head starts on the leftmost space of the tape.

As the Turing machine computes, the state, tape contents, and head location changes. A combination of these three items is a \vocab{configuration} of the Turing machine. We represent them in a specific way. If the current state is \(q\), the tape contains \(uv \in \Gamma^*\), and the tape head is on the first character of \(v\), then we say that the configuration is \(uqv\).

We will say that a configuration \(C_1\) yields \(C_2\) if the Turing machine can go from \(C_1\) to \(C_2\) in one step. More formally, for all \(u, v \in \Gamma^*\), if for some \(a, b, c \in \Gamma\) we have \(\delta(q_i, b) = (q_j, c, \mathrm L)\), then
\[
	ua q_i bv \quad\text{yields}\quad u q_i acv,
\]
and if \(\delta(q_i, b) = (q_j, c, \mathrm R)\), then
\[
	ua q_i bv \quad\text{yields}\quad uac q_j v.
\]
If the head is at the left end, then receiving an \(\mathrm L\) will not move the head.

The \vocab{starting configuration} of \(M\) on input \(w\) is the configuration \(q_0 w\). An \vocab{accepting configuration} is one where the state is \(q_{\text{accept}}\), and a \vocab{rejecting configuration} is one where the state is \(q_{\text{reject}}\). We call accepting and rejecting configurations \vocab{halting configurations}, and do not yield any configurations. If the machine never halts, then we say that it \vocab{loops}.

Then, we say that \(M\) accepts a string \(w\) if there exists configurations \(C_1, \dots, C_k\) such that \(C_1 = q_0 w\), \(C_i\) yields \(C_{i + 1}\), and \(C_k\) is an accepting configuration. Then the \vocab{language of \(M\)} is
\[
	L(M) \coloneqq \set{w \in \Sigma^* \mid \text{$M$ accepts $w$}}.
\]
We also say that \(M\) \vocab{recognizes} \(L(M)\).





\end{document}

