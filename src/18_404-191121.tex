  \documentclass{standalone}
  \usepackage{chez}

  \begin{document}
  %\chapter{November 21, 2019}
  \section{Intractable Problems}
  Recall that we have
  \[
    \mathsf{L} \subseteq
    \mathsf{NL} \subseteq
    \mathsf{P} \subseteq
    \mathsf{NP} \subseteq
    \mathsf{PSPACE},
  \]
  where \(\mathsf{NL} \neq \mathsf{PSPACE}\).
  The problems of whether \(\mathsf{L} \overset{?}{=} \mathsf{P}\) and
  \(\mathsf{P} \overset{?}{=} \mathsf{PSPACE}\) are open,
  but \(\mathsf{NL} \neq \mathsf{PSPACE}\) implies that
  at least one of them is false.

  Recall that \(\textit{EQ}_{\textsf{DFA}}\) is decidable,
  and using the same \textsf{TM},
  we have that \(\textit{EQ}_{\textsf{DFA}} \in \mathsf{P}\).
  We can also show that \(\textit{EQ}_{\textsf{NFA}} \in \mathsf{PSPACE}\).

  \begin{proposition}
    \(\textit{EQ}_{\textsf{NFA}} \in \mathsf{PSPACE}\).
  \end{proposition}
  \begin{proof}
    Since \(\mathsf{PSPACE}\) is deterministic,
    \(A \in \mathsf{PSPACE} \iff \ol A \in \mathsf{PSPACE}\).
    We proceed by showing that
    \(\ol{\textit{EQ}_{\mathsf{NFA}}} \in \mathsf{NPSPACE = \mathsf{PSPACE}}\).
    The idea is to nondeterministically guess the string
    that yields different results for the two \textsf{NFA}s.
    It is clear that this runs in polynomial space.
  \end{proof}

  Similarly, we have a similar theorem for regular expressions:
  \begin{corollary}
    \(\textit{EQ}_{\textsf{REGEX}} \in \mathsf{PSPACE}\).
  \end{corollary}
  \begin{proof}
    This is immediately because the conversion from
    \(\textsf{REGEX} \to \textsf{NFA}\) can be done in polynomial space.
  \end{proof}


  \subsection{Exponential Space}
  Consider a modified version of regular expressions that has exponentiation,
  i.e.
  \[
    R^i = \underbracket{RR \dots R}_{\text{$i$ times}}.
  \]
  Let a regular expression with this feature be called an \(\REGEXup\).

  \begin{definition}
    Let the languages decidable in exponential time and space be, respectively
    \begin{gather*}
      \mathsf{EXPTIME} = \bigunion_k \TIME(2^{n^k}) \\
      \mathsf{EXPSPACE} = \bigunion_k \SPACE(2^{n^k}).
    \end{gather*}
  \end{definition}

  \begin{definition}
    We say that a language \(B\) is \vocab{\(\mathsf{EXPSPACE}\)-complete} if
    \begin{itemize}
      \item \(B \in \mathsf{EXPSPACE}\) and
      \item for all \(A \in \mathsf{EXPSPACE}\),
            we have \(A \polyredu B\),
            i.e.\ is \vocab{\(\mathsf{EXPSPACE}\)-hard}.
  \end{itemize}
\end{definition}

\begin{theorem}
  \(\textit{EQ}_{\REGEXup} = \set{\angles{R_1, R_2} \mid \text{%
    $R_1$ and $R_2$ are $\REGEXup$ such that $L(R_1) = L(R_2)$%
  }}\) is \(\mathsf{EXPSPACE}\)-complete.
\end{theorem}
\begin{proof}
  We will show that \(\textit{EQ}_{\REGEXup}\)
  is \(\mathsf{EXPSPACE}\)-complete.
  It is clear that \(\textit{EQ}_{\REGEXup} \in \mathsf{EXPSPACE}\)
  because \(\textit{EQ}_{\textsf{REGEX}} \in \mathsf{PSPACE}\).
  In particular, we can expand the \(\REGEXup\) into a \textsf{REGEX}
  in exponential space, and then use the \(\mathsf{PSPACE}\) algorithm for
  \(\textit{EQ}_{\textsf{REGEX}}\) on the exponentially sized \textsf{REGEX}s.

  Now consider some \(A \in \mathsf{EXPSPACE}\) decided by a \textsf{TM} \(M\)
  that runs in space \(O(2^{n^k})\).
  Consider the reduction mapping \(w \mapsto \angles{R_1, R_2}\)
  where \(R_1\) is a \(\REGEXup\) that generates all strings,
  and \(R_2\) is a \(\REGEXup\) that generates all strings
  except for the rejecting computation history of \(M\) on \(w\).
  Then \(M\) accepts \(w\) if and only if \(L(R_1) = L(R_2)\), as desired.

  To actually construct \(R_2\), we can just check to make sure
  that the computation history makes a mistake.
  Let \(\Delta = \Gamma \union Q \union \set{\texttt{\#}}\)
  be the alphabet for the configurations.
  Consider the configuration history
  \[
    c_1 \texttt{\#} c_2 \texttt{\#} \dots \texttt{\#} c_{\text{reg}},
  \]
  where we pad all configurations at the end with \(\blank\) for convenience.
  Then let
  \[
    R_2 = R_{\text{bad-start}} \union R_{\text{bad-move}}
                               \union R_{\text{bad-reject}},
  \]
  where the parts are defined as follows
  (where we adopt the notation \(X - a \coloneqq X \setminus \set{a}\)):
  \begin{itemize}
    \item Let
          \(R_{\text{bad-start}} =
            S_0 \union S_1 \union S_2 \union S_n \union S_{\text{blanks}}\),
          where \(S_i\) means that the \(i\)th character is incorrect.
          In particular,
          \[
            S_i = \Delta^i (\Delta - w_i) \Delta^*
              \qquad \text{and} \qquad
              S_{\text{blanks}} = \Delta^{n + 1}
                                  (\Delta - \eps)^{2^{n^k} - n - 2}
                                  (\Delta - \blank) \Delta^*.
          \]
          Note that we have used the exponentiation function of
          the \(\REGEXup\) so that \(R_{\text{bad-start}}\) is not
          exponentially long.

    \item Similarly, let \(R_{\text{bad-reject}} = (\Delta - q_{\text{reject}})^*\).

    \item For \(R_{\text{bad-move}}\),
          we can use the same idea as we did in Cook-Levin.
          We look at three consecutive symbols in a configuration
          as well as the configuration after it,
          and we reject the sets of symbols
          that do not correspond to a legal move.
          In particular,
    \[
      R_{\text{bad-move}} = \bigunion_{\text{bad $(abc, def)$}} \Delta^* abc \Delta^{2^{n^k} - 2} def \Delta^*.
    \]
  \end{itemize}

  Note that \(R_2\) is polynomial in length, as desired.
\end{proof}






\end{document}

