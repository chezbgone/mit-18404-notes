\documentclass{standalone}
\usepackage{chez}

\begin{document}
%\chapter{December 10, 2019}
\section{\texorpdfstring{\(\texttt\#\textit{SAT} \in \mathsf{IP}\)}{\#SAT in IP}}
Consider the problem
\[
	\textsf\#\textit{SAT} = \set{
    \angles{\phi, k} \mid \text{$\phi$ has exactly $k$ satisfying assignments}
  }.
\]
For a formula \(\phi\) on variables \(x_1, \dots, x_m\) and
\(a_1, \dots, a_i \in \set{\mathtt 0, \mathtt 1}\),
let \(\numphi(a_1, \dots, a_i)\) be the number of satisfying assignments
with presets \(x_1 = a_1\), \(x_2 = a_2\), \ldots, \(x_i = a_i\).
Then,
\begin{itemize}
	\item \(\numphi(a_1, \dots, a_i) = \numphi(a_1, \dots, a_i, \mathtt 0) +
                                     \numphi(a_1, \dots, a_i, \mathtt 1)\), and
	\item \(\numphi(a_1, \dots, a_m) = \phi(a_1, \dots, a_m)\).
\end{itemize}

There is a straightforward exponential protocol for \(\texttt\#\textit{SAT}\).
Suppose that \(P\) claims that \(\angles{\phi, k} \in \texttt\#\textit{SAT}\).
\begin{enumerate}[start=0]
	\item \(P\) sends \(\numphi()\), and \(V\) checks \(k = \numphi()\).
	\item \(P\) sends \(\numphi(\mathtt 0), \numphi(\mathtt 1)\),
    and \(V\) checks \(\numphi() = \numphi(\mathtt 0) + \numphi(\mathtt 1)\).
	\item \(P\) sends \(\numphi(\mathtt{00}),
                      \numphi(\mathtt{01}),
                      \numphi(\mathtt{10}),
                      \numphi(\mathtt{11})\),
    and \(V\) checks \(\numphi(\mathtt 0) = \numphi(\mathtt{00}) +
                                            \numphi(\mathtt{01})\)
                 and \(\numphi(\mathtt 1) = \numphi(\mathtt{10}) +
                                            \numphi(\mathtt{11})\).
	\item[\(\vdots\)]
	\item[\(m\).] \(P\) sends \(\numphi(\mathtt{0}^m), \dots,
                              \numphi(\mathtt 1^m)\), and \(V\) checks
    \(\numphi(\mathtt 0^{m - 1}) = \numphi(\mathtt 0^m) +
                                   \numphi(\mathtt 0^{m - 1} \mathtt 1), \dots,
      \numphi(\mathtt 1^{m - 1}) = \numphi(\mathtt 1^{m - 1} \mathtt 0) +
                                   \numphi(\mathtt 1^m)\).
	\item[\(m + 1\).] \(V\) checks \(\numphi(\mathtt 0^m) = \phi(\mathtt 0^m),
                            \dots, \numphi(\mathtt 1^m) \phi(\mathtt 1^m)\).
\end{enumerate}

The problem is that at every step, we are doubling the number of things that
the prover is sending, because we have to extend the preset variables
by both \(\mathtt 0\) and \(\mathtt 1\).
To fix this, the key idea is to extend by a non-boolean value instead.
We will use the standard arithmetization:
\begin{itemize}[nosep]
	\item \(x \land y \to xy\)
	\item \(\ol x \to 1 - x\)
	\item \(x \lor y \to x + y - xy\)
	\item \(\phi \to \text{some polynomial $P_\phi$ with degree
    less than $\size \phi$}\)
\end{itemize}
Moreover, we can extend \(\numphi\) to be
\[
	\numphi(a_1, \dots, a_i) =
    \sum_{\mathclap{a_{i + 1}, \dots, a_{m} \in \set{\mathtt 0, \mathtt 1}}}
      P_\phi(a_1, \dots, a_i, a_{i + 1}, \dots, a_{m})
\]
over \(a_1, \dots, a_i \in \FF_q\).
Note that we still have
\begin{itemize}
	\item \(\numphi(a_1, \dots, a_i) = \numphi(a_1, \dots, a_i, \mathtt 0) +
                                     \numphi(a_1, \dots, a_i, \mathtt 1)\), and
	\item \(\numphi(a_1, \dots, a_m) = \phi(a_1, \dots, a_m)\).
\end{itemize}
Now consider the new protocol:
\begin{enumerate}[start=0, nosep]
	\item \(P\) sends \(\numphi()\) and \(V\) checks \(k = \numphi()\).

	\item \begin{enumerate}
		\item \(P\) sends \(\numphi(z)\) as a polynomial and
          \(V\) checks \(\numphi() = \numphi(\texttt 0) + \numphi(\texttt 1)\).
          \(V\) can get \(\numphi(\mathtt 0)\) and \(\numphi(\mathtt 1)\)
          by plugging into \(\numphi(z)\).

		\item \(V\) requests a random \(r_1 \in \FF_q\) and
          \(P\) sends \(\numphi(r_1)\).
	\end{enumerate}

	\item \begin{enumerate}
		\item \(P\) sends \(\numphi(r_1, z)\) and
          \(V\) checks \(\numphi(r_1) = \numphi(r_1, \texttt 0) +
                                        \numphi(r_1, \texttt 1)\).

		\item \(V\) requests a random \(r_2 \in \FF_q\) and
          \(P\) sends \(\numphi(r_1, r_2)\).
	\end{enumerate}

	\item[\(\vdots\)]

	\item[\(m\).] \begin{enumerate}
		\item \(P\) sends \(\numphi(r_1, \dots, r_{m - 1}, z)\) and
          \(V\) checks \(\numphi(r_1, \dots, r_{m - 1}) =
                \numphi(r_1, \dots, r_{m - 1}, \texttt 0) +
                \numphi(r_1, \dots, r_{m - 1}, \texttt 1)\).

		\item \(V\) requests a random \(r_m \in \FF_q\) and
          \(P\) sends \(\numphi(r_1, \dots, r_m)\).
	\end{enumerate}

	\item[\(m + 1\).] \(V\) checks
                    \(\numphi(r_1, \dots, r_m) = P_{\phi(r_1, \dots, r_m)}\)
                    and \textsc{accept}s if true.
\end{enumerate}




\end{document}

