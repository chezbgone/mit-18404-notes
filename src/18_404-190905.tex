\documentclass{standalone}
\usepackage{chez}

\begin{document}
The class website is \href{https://math.mit.edu/~sipser/18404/}{\texttt{here}}.

\section{Finite Automata}
We're going to start with a discussion of computability, in which we need models of computation.

Let's start with an object \(M_1\) that is represented by the following diagram:
\begin{center}
	\begin{tikzpicture}[automaton, node distance=1.7cm]
		\node[state, initial]   (s_1) {\(q_1\)};
		\node[state]            (s_2) [right = of s_1] {\(q_2\)};
		\node[state, accepting] (s_3) [right = of s_2] {\(q_3\)};

		\path (s_1) edge [loop above] node {\(\texttt 0\)} (s_1);
		\path (s_1) edge [bend left]  node {\(\texttt 1\)} (s_2);
		\path (s_2) edge [bend left]  node {\(\texttt 0\)} (s_1);
		\path (s_2) edge []           node {\(\texttt 1\)} (s_3);
		\path (s_3) edge [loop right] node {\(\texttt 0, \texttt 1\)} (s_3);
	\end{tikzpicture}
\end{center}
We have the states \(q_1\), \(q_2\), and \(q_3\), transitions represented by the arrows and their labels, the start label labeled with an incoming arrow, and the accepting state with a double outline.

The inputs are strings, and to determine whether a string is accepted or rejected, we begin at the start input, and follow arrows corresponding to each character until reach the end of the string. If the ending state is an accepting state, we say that the input is accepted, and otherwise, the input is rejected.

\begin{example}
	In \(M_1\), \texttt{0101} is rejected and \texttt{01101} is accepted.
\end{example}

For a finite automata \(M\), we can think about all of the strings that \(M\) accepts:
\[
	\set{\text{strings \(w\)} \mid \text{\(M\) accepts \(w\)}}.
\]
We usually call this the language of \(M\), and we also sometimes say that \(M\) recognizes \(A\).

\subsection{More Rigorously}
\begin{definition}[Finite Automaton]
	A \vocab{finite automaton} \(M\) is a tuple \((Q, \Sigma, \delta, q_0, F)\) where
	\begin{itemize}[nosep]
		\item \(Q\) is a finite set of states,
		\item \(\Sigma\) is a set of symbols, called the \vocab{alphabet},
		\item \(\delta \colon Q \times \Sigma \to Q\) is the transition function, i.e. the arrows in the diagram,
		\item \(q_0 \in Q\) is the start state, and
		\item \(F \subseteq Q\) are the accepting states.
	\end{itemize}
\end{definition}

\begin{remark}
	Since \(\delta\) has a codomain of \(Q\), this is a deterministic finite automata, i.e. there is only one possible ``next'' output given a state and a symbol.
\end{remark}

\begin{definition}
	We say that a finite automaton \(M = (Q, \Sigma, \delta, q_0, F)\) \vocab{accepts} a string (a finite sequence of symbols in \(\Sigma\)) \(w = w_1 w_2 \dots w_n\), where \(w_i \in \Sigma\) if there exists a sequence of states \(r_0, r_1, \dots, r_n\) where
	\begin{itemize}[nosep]
		\item \(r_0 = q_0\),
		\item \(r_n \in F\), and
		\item \(\delta(r_{i - 1}, w_i) = r_i\) for \(1 \leq i \leq n\).
	\end{itemize}
\end{definition}

\begin{definition}
	The \vocab{language} of \(M\) is \(L(M) = \set{w \mid \text{$M$ accepts $w$}}\). Then, we say that \(M\) \vocab{recognizes} \(L(M)\).

	If \(A = L(M)\) for some finite automaton \(M\), then \(A\) is a \vocab{regular} language.
\end{definition}

\begin{example}[Regular Languages]
	\begin{itemize}[nosep]
		\item \(A = \set{w \mid \text{\texttt{11} is a substring of $w$}}\) is regular.
		\item \(B = \set{w \mid \text{$w$ starts and ends with the same symbol}}\) is regular.
		\item \(C = \set{w \mid \text{$w$ has an equal number of \texttt{0}s and \texttt{1}s}}\) is regular.
	\end{itemize}
\end{example}

\subsection{Operations}
We can also define operations on languages. Let \(A\) and \(B\) be languages. Then, the \vocab{union} of \(A\) and \(B\) is
\[
	A \union B = \set{w \mid \text{$w \in A$ or $w \in B$}}.
\]
The \vocab{concatenation} of \(A\) and \(B\) is
\[
	A \circ B = \set{xy \mid \text{$x \in A$ and $y \in B$}}.
\]
We will often omit the circle for convenience. The \vocab{star} operator gives
\[
	A^* = \set{x_1 x_2 \dots x_k \mid x_i \in A, k \in \NN}.
\]
Note that the empty string \(\eps\) is in \(A^*\) from the case \(k = 0\).

\begin{proposition}<union_of_regular_is_regular>
	If \(A\) and \(B\) are regular, then so is \(A \union B\).
\end{proposition}
\begin{proof}
	First we'll show that \(A \union B\) is regular. Suppose \(M_A = (Q_A, \Sigma, \delta_A, q_A, F_A)\) and \(M_B = (Q_B, \Sigma, \delta_B, q_B, F_B)\) recognize \(A\) and \(B\) respectively. We assume that the alphabets are the same, but the proof would be similar albeit messier if they were not the same. The idea is that we can ``run both of the machines at the same time''. In particular, we will keep track of the states of each machine in ordered pairs.

	Our desired finite automaton that accepts \(A \union B\) is \((Q, \Sigma, \delta, q, F)\), where
	\begin{itemize}[nosep]
		\item \(Q = Q_A \times Q_B\),
		\item \(\delta \colon ((a, b), \sigma) \mapsto (\delta_A(a, \sigma), \delta_B(b, \sigma))\),
		\item \(q = (q_A, q_B)\), and
		\item \(F = \set{(a, b) \in Q_A \times Q_B \mid \text{$a \in F_A$ or $b \in F_B$}} = (F_A \times Q_B) \union (Q_A \times F_B)\). \qedhere
	\end{itemize}
\end{proof}

\begin{remark}
	Note that we can prove that \(A \intersect B \coloneqq \set{w \mid \text{$w \in A$ and $w \in B$}}\) is regular with the exact same machine as \(A \union B\) except we replace the final states with \(F_A \times F_B\).
\end{remark}



\end{document}

